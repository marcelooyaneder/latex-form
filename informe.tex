\documentclass[10pt]{article}

%--------------------BLOQUE DE FORMATO-------------
\usepackage[letterpaper,left=3cm,top=2.5cm,right=3cm,bottom=2.5cm]{geometry} %margenes.
\usepackage{amsmath,amsthm,amsfonts,amssymb,amscd} 
\usepackage[utf8]{inputenc} %carácteres acentuados.
\usepackage{enumerate}
\usepackage{mathrsfs}
\usepackage{xcolor}
\usepackage{graphicx} %paquete para insertar graficos e imagenes.
\usepackage{listings} %paquete para ingresar codigos e.g. MATLAB, Python, Fortran.
\usepackage{color}
\usepackage{bm}
\usepackage{nicefrac} %Para las fracciones
\usepackage{booktabs}
\usepackage{url}

\usepackage{parskip}
\setlength{\parindent}{0cm}

\usepackage[spanish,mexico]{babel} % Utilización de reglas foneticas para español mexicano.

%------------PARA AJUSTAR FUENTES-----------------------------
\usepackage[T1]{fontenc}%habilita caracteres no latinos.
\ifxetexorluatex
  % Para compilar con xetex o lualatex.
  \setmainfont{Arial}
  \usepackage{arevmath}
\else
  % compilar con pdflatex
  \usepackage[scaled]{helvet}%fuente arial=helvet.
  \usepackage{sfmath} %paquete para ajustar la fuente matematica a serif.  
  %\usepackage{mathastext} %formulas matematicas, de acuerdo a la fuente elegida.
\fi
\renewcommand*\familydefault{\sfdefault}%fuente
\linespread{1.5} % Interlineado
%------------------------------------------------------------------------------

\usepackage{fancyhdr} % Modificar encabezados y pies de paginas.
\fancyhead{} % Deja el encabezado en blanco.
\fancyfoot{} % Deja el pie de pagina en blanco.
\renewcommand{\headrulewidth}{0pt} %eliminar regla del encabezado.
\rfoot{\thepage} % ubicar la numeración de paginas a la derecha.

\usepackage{titlesec} %modifición de secciones
\titleformat{\section}[hang]{\normalfont\bfseries\centering}{\thesection}{5pt.}{} %modificación secciones.
\titleformat{\subsection}[hang]{\normalfont\bfseries\centering}{\thesubsection}{5pt}{} %modificación subsecciones.
\titleformat{\subsubsection}[hang]{\normalfont\bfseries}{\thesubsubsection}{5pt}{} %modificación subsecciones.

\usepackage{multirow} %para combinar en tablas, usado para la portada.
\usepackage{float} %para el float H en tablas.
\usepackage{blindtext} % paquete para generar texto de prueba en el informe.

\renewcommand{\theequation}{\arabic{section}.\arabic{equation}} %definición enumeración de ecuaciones.
% para hacer referencia a ecuaciones utilizar \eqref{•}

%reedifinición para hacer enumeración por secciones y subsecciones.
\renewcommand{\labelenumi}{\arabic{section}.\arabic{enumi}}
\renewcommand{\labelenumii}{\arabic{section}.\arabic{enumi}.\arabic{enumii}}
\renewcommand{\labelenumiii}{\arabic{section}.\arabic{enumi}.\arabic{enumii}.\arabic{enumiii}}
\renewcommand{\labelenumiv}{\arabic{section}.\arabic{enumi}.\arabic{enumii}.\arabic{enumiii}.\arabic{enumiv}}

\renewcommand{\thefigure}{\arabic{section}.\arabic{figure}} %definición enumeración de ecuaciones.
\renewcommand{\thetable}{\arabic{section}.\arabic{table}} %definición enumeración de tablas.


%----------New MathCommands by Luis Henríquez    
\newcommand{\der}[2]{\frac{\partial #1}{\partial #2}}
\newcommand{\dero}[2]{\frac{d #1}{d #2}}
\newcommand{\lapl}[4]{\der{}{#1} \p{#2 \der{#3}{#1}}_{#4}}
\newcommand{\p}[1]{\left( #1 \right)} %Parentheses
\newcommand{\ps}[1]{\left[ #1 \right]} %Square
\newcommand{\pc}[1]{\left \{ #1 \right \}} %Curve
\newcommand{\pa}[1]{\left \langle #1 \right \rangle} %Angle


\newcommand{\blue}[1]{{\color{blue}#1}}
\newcommand{\red}[1]{{\color{red}#1}}

\usepackage{etoolbox}
\patchcmd{\thebibliography}{\section*{\refname}}{}{}{}

%\footcite{} se utiliza para agregar citas bibliográficas al píe de pagina y en la pagina de referencias

%-------------------------------FIN BLOQUE DE INSTRUCCIONES


%para hacer referencia a ecuaciones utilizar \eqref
%para hacer citas al píe de pagina y referencias utilizar \footcite{•}

\begin{document}
%------------------------------------------------------------------------------------------------------------------
%-------------------------------PORTADA-------------------------------------------------------------------

\begin{flushleft}
    \begin{tabular}{ll} %NEED \usepackage{multirow}
      \multirow{4}{2cm}{\includegraphics[height=0.1\textheight]{imagenes/logo_diq.jpeg} }&UNIVERSIDAD DE SANTIAGO DE CHILE  \\
      &FACULTAD DE INGENIERÍA\\
      &DEPARTAMENTO DE INGENIERÍA QUÍMICA \\
    \end{tabular}
\end{flushleft}


\begin{center}
\vspace*{8cm}
 
\textbf{INFORME N$\bm{^{\circ}}$ 1}\\ %TITULO DE LA EXPERIENCIA
\textbf{NOMBRE DE LA EXPERIENCIA} 
       %\vspace{0.5cm}
        %Thesis Subtitle
\vfill 
% \vspace{0.8cm}
\end{center}
   
\begin{flushright}
   \begin{tabular}{lll}
   CURSO       &:& Nombre del Curso.\\
   INTEGRANTES &:& Juan Pérez.\\
               & & Juan Pérez.\\
               & & Juan Pérez.\\
   PROFESOR    &:& Profe Sor.\\
   AYUDANTE    &:& Ayud Ante.\\
   FECHA EXPERIENCIA &:& 18 de Junio de 2019.\\
   FECHA ENTREGA     &:& 19 de Junio de 2019.\\
   \end{tabular}
\end{flushright}
\end{titlepage}

%----------------------------------------------------------------------------------------
%----------------------------------------RESUMEN--------------------------------
\pagestyle{empty} %suprimir numeración 
%----------------------------------------------------------------------------------------
\section*{RESUMEN}

\input{secciones/resumen}

\blindtext %texto para prueba
%-----------------------------------------------------------------------------------------
%----------------------------------ÍNDICE.-----------------------------------
%\newpage 
%\tableofcontents %%FALTA CORREGIR
%\newpage
%\newrgbcolor{xdxdff}{0.49 0.49 1}
%\newrgbcolor{ffqqtt}{1 0 0.2}
%\psset{xunit=1.0cm,yunit=1.0cm,algebraic=true,dotstyle=*,dotsize=3pt 0,linewidth=0.8pt,arrowsize=3pt 2,arrowinset=0.25}




%-----------------------------------------------------------------------------------------
%----------------------------------OBJETIVOS.-----------------------------------
\newpage
\pagestyle{fancy}
\setcounter{page}{2} %empezar numeración desde está pagina
\setcounter{equation}{0} %empezar numeración desde 1 para ecuaciones
\setcounter{figure}{0} %empezar numeración desde 1 para figuras
\setcounter{table}{0} %empezar numeración desde 1 para tablas
%----------------------------------------------------------------------------------------
\section{OBJETIVOS}

\input{secciones/objetivos}

\blindmathpaper %texto para prueba
%-----------------------------------------------------------------------------------------
%----------------------------------MARCO TEÓRICO.-----------------------------------
\newpage
\setcounter{equation}{0} %empezar numeración desde 1 para ecuaciones
\setcounter{figure}{0} %empezar numeración desde 1 para figuras
\setcounter{table}{0} %empezar numeración desde 1 para tablas
%----------------------------------------------------------------------------------------
\section{MARCO TEÓRICO}

\input{secciones/marcoteorico}

\blindmathpaper %texto para prueba

%------------------------------------------------------------------------------------------
%---------------------APARATOS Y ACCESORIOS--------------------------
\newpage
\setcounter{equation}{0} %empezar numeración desde 1 para ecuaciones
\setcounter{figure}{0} %empezar numeración desde 1 para figuras
\setcounter{table}{0} %empezar numeración desde 1 para tablas
%----------------------------------------------------------------------------------------
\section{APARATOS Y ACCESORIOS}

\input{secciones/aparatosyaccesorios}

\blindmathpaper %texto para prueba


%-------------------------------------------------------------------------------------------
%------------------------------PROCEDIMIENTO EXPERIMENTAL
\newpage
\setcounter{equation}{0} %empezar numeración desde 1 para ecuaciones
\setcounter{figure}{0} %empezar numeración desde 1 para figuras
\setcounter{table}{0} %empezar numeración desde 1 para tablas
%----------------------------------------------------------------------------------------
\section{PROCEDIMIENTO EXPERIMENTAL}

\input{secciones/procedimientoexperimental}

\blindmathpaper %texto para prueba


%-------------------------------------------------------------------------------------------
%-----------------------------------------DATOS----------------------------------------
\newpage
\setcounter{equation}{0} %empezar numeración desde 1 para ecuaciones
\setcounter{figure}{0} %empezar numeración desde 1 para figuras
\setcounter{table}{0} %empezar numeración desde 1 para tablas
%----------------------------------------------------------------------------------------
\section{DATOS}

\input{secciones/datos}

\blindmathpaper %texto para prueba


%-------------------------------------------------------------------------------------------
%----------------RESULTADOS Y DISCUSIONES----------------------------
\newpage
\setcounter{equation}{0} %empezar numeración desde 1 para ecuaciones
\setcounter{figure}{0} %empezar numeración desde 1 para figuras
\setcounter{table}{0} %empezar numeración desde 1 para tablas
%----------------------------------------------------------------------------------------
\section{RESULTADOS Y DISCUSIONES}

\input{secciones/resultadosydiscusiones}

\blindmathpaper %texto para prueba


%-------------------------------------------------------------------------------------------
%-----------------------------CONCLUSIONES------------------------------------
\newpage
\setcounter{equation}{0} %empezar numeración desde 1 para ecuaciones
\setcounter{figure}{0} %empezar numeración desde 1 para figuras
\setcounter{table}{0} %empezar numeración desde 1 para tablas
%----------------------------------------------------------------------------------------
\section{CONCLUSIONES}

\input{secciones/conclusiones}

\blindmathpaper %texto para prueba


%----------------------------------------------------------------------------------------
%----------------------------REFERENCE LIST--------------------------------
\newpage
\setcounter{equation}{0} %empezar numeración desde 1 para ecuaciones
\setcounter{figure}{0} %empezar numeración desde 1 para figuras
\setcounter{table}{0} %empezar numeración desde 1 para tablas
%----------------------------------------------------------------------------------------
\section{BIBLIOGRAFÍA}
\bibliography{bibliografia} 
\bibliographystyle{apalike} 
\

%-------------------------------------------------------------------------------------------
%-----------------------------APENDICES-------------------------------------------
\newpage
\pagestyle{empty} %suprimir numeración 
\setcounter{equation}{0} %empezar numeración desde 1 para ecuaciones, AGREGAR ESTO ANTES DE CADA SUBSECCIÓN PARA COMENZAR CUENTA DESDE 1
\setcounter{figure}{0} %empezar numeración desde 1 para figuras
\setcounter{table}{0} %empezar numeración desde 1 para tablas
\renewcommand{\thesubsection}{APÉNDICE \Alph{subsection}:}                            %REENUMERACION DE SUBSECCIONES
\renewcommand{\thesubsubsection}{\Alph{subsection}.\arabic{subsubsection}:}       %REENUMERACION DE SUBSUBSECCIONES
\renewcommand{\theequation}{.\Alph{subsection}\arabic{equation}} %definición enumeración de ecuaciones
\renewcommand{\thefigure}{\Alph{subsection}.\arabic{figure}} %definición enumeración de ecuaciones
\renewcommand{\thetable}{\Alph{subsection}.\arabic{table}} %definición enumeración de tablas

%----------------------------------------------------------------------------------------
\section*{APÉNDICES}

\input{secciones/append}

\end{document}
